% Los paquetes que usa
\usepackage{ifthen}
\usepackage{color}

%%%%%%%%%%%%%%%%%%%%%%%%%%%%
%%% INSTRUCCIONES DE USO %%%
%%%%%%%%%%%%%%%%%%%%%%%%%%%%
% Para incluir esas macros poner % Los paquetes que usa
\usepackage{ifthen}
\usepackage{color}

%%%%%%%%%%%%%%%%%%%%%%%%%%%%
%%% INSTRUCCIONES DE USO %%%
%%%%%%%%%%%%%%%%%%%%%%%%%%%%
% Para incluir esas macros poner % Los paquetes que usa
\usepackage{ifthen}
\usepackage{color}

%%%%%%%%%%%%%%%%%%%%%%%%%%%%
%%% INSTRUCCIONES DE USO %%%
%%%%%%%%%%%%%%%%%%%%%%%%%%%%
% Para incluir esas macros poner % Los paquetes que usa
\usepackage{ifthen}
\usepackage{color}

%%%%%%%%%%%%%%%%%%%%%%%%%%%%
%%% INSTRUCCIONES DE USO %%%
%%%%%%%%%%%%%%%%%%%%%%%%%%%%
% Para incluir esas macros poner \input{designMacros} junto con los \usepackage
%
% EL ENVIRONMENT INTERFAZ
% Se usa con \begin{interfaz} y \end{interfaz}.
% Toma 4 par�metros:
%  - Nombre de la funci�n
%  - Par�metros
%  - Tipo del resultado (Opcional)
%  - Complejidad (Oppcional)
% 
% Para las funciones sobre estr se puede poner \i antes del nombre de la funci�n para una i m�s pro. Eg: \begin{interfaz}{\i Desencolar} etc.
% Para poner pre y postcondiciones se usan las macros \pre y \post (No hace falta poner math mode para usar s�mbolos)
% Se puede agregar un comentario usando \nota{el comentario}
% Para referirse al resultado se puede usar \res en vez de escribir "resultado"
%
% EJEMPLO 
% \begin{interfaz}{encolar}{inout $c$: cola(int), in $elem$: int}{}{\Ode{1}}
%	 \pre{c = c_0}
%	 \post{c = encolar(c_0, e)}
% \end{interfaz}
% 
% 


%%%%%%%%%%%%%%%%%%%%%
%%% IMPLEMENTACI�N %%%
%%%%%%%%%%%%%%%%%%%%%

%%%Contadores usados%%%
\newcounter{linenumber}
\newcounter{showln}
\newcounter{tabs}
\newcounter{backtabs}

%%%Envireonment para interfaces%%%
\newenvironment{interfaz}[4]{

	% Primero definamos \pre y \post
	\newcommand{\pre}[1]{\ensuremath{\left\{\begin{array}{l}##1\end{array}\right\}}\par}
	\newcommand{\post}[1]{\pre{##1}}

	% Para comentarios
	\newcommand{\nota}[1]{\begin{small}\textbf{Nota:} ##1\end{small}}

	% El prefijo i de las funciones imperativas
	\renewcommand{\i}{\textit{i}}

	% Por si alg�n d�a dejamos de usar resultado.
	\newcommand{\res}{resultado}


  % Aridad de la funci�n
  \noindent
  	\textsl{#1}(#2) 	
  	\ifthenelse{\equal{#3}{}}{}{ $\rightarrow$ \textit{\res}: #3}  	
  	\ifthenelse{\equal{#4}{}}{}{\hspace{1 ex} #4}
  	
  % Reducimos un touch el indent
  \setlength{\parindent}{1 ex}
  \par
}{
	% Un espacio en blanco al final
	\vspace{3 ex}
} junto con los \usepackage
%
% EL ENVIRONMENT INTERFAZ
% Se usa con \begin{interfaz} y \end{interfaz}.
% Toma 4 par�metros:
%  - Nombre de la funci�n
%  - Par�metros
%  - Tipo del resultado (Opcional)
%  - Complejidad (Oppcional)
% 
% Para las funciones sobre estr se puede poner \i antes del nombre de la funci�n para una i m�s pro. Eg: \begin{interfaz}{\i Desencolar} etc.
% Para poner pre y postcondiciones se usan las macros \pre y \post (No hace falta poner math mode para usar s�mbolos)
% Se puede agregar un comentario usando \nota{el comentario}
% Para referirse al resultado se puede usar \res en vez de escribir "resultado"
%
% EJEMPLO 
% \begin{interfaz}{encolar}{inout $c$: cola(int), in $elem$: int}{}{\Ode{1}}
%	 \pre{c = c_0}
%	 \post{c = encolar(c_0, e)}
% \end{interfaz}
% 
% 


%%%%%%%%%%%%%%%%%%%%%
%%% IMPLEMENTACI�N %%%
%%%%%%%%%%%%%%%%%%%%%

%%%Contadores usados%%%
\newcounter{linenumber}
\newcounter{showln}
\newcounter{tabs}
\newcounter{backtabs}

%%%Envireonment para interfaces%%%
\newenvironment{interfaz}[4]{

	% Primero definamos \pre y \post
	\newcommand{\pre}[1]{\ensuremath{\left\{\begin{array}{l}##1\end{array}\right\}}\par}
	\newcommand{\post}[1]{\pre{##1}}

	% Para comentarios
	\newcommand{\nota}[1]{\begin{small}\textbf{Nota:} ##1\end{small}}

	% El prefijo i de las funciones imperativas
	\renewcommand{\i}{\textit{i}}

	% Por si alg�n d�a dejamos de usar resultado.
	\newcommand{\res}{resultado}


  % Aridad de la funci�n
  \noindent
  	\textsl{#1}(#2) 	
  	\ifthenelse{\equal{#3}{}}{}{ $\rightarrow$ \textit{\res}: #3}  	
  	\ifthenelse{\equal{#4}{}}{}{\hspace{1 ex} #4}
  	
  % Reducimos un touch el indent
  \setlength{\parindent}{1 ex}
  \par
}{
	% Un espacio en blanco al final
	\vspace{3 ex}
} junto con los \usepackage
%
% EL ENVIRONMENT INTERFAZ
% Se usa con \begin{interfaz} y \end{interfaz}.
% Toma 4 par�metros:
%  - Nombre de la funci�n
%  - Par�metros
%  - Tipo del resultado (Opcional)
%  - Complejidad (Oppcional)
% 
% Para las funciones sobre estr se puede poner \i antes del nombre de la funci�n para una i m�s pro. Eg: \begin{interfaz}{\i Desencolar} etc.
% Para poner pre y postcondiciones se usan las macros \pre y \post (No hace falta poner math mode para usar s�mbolos)
% Se puede agregar un comentario usando \nota{el comentario}
% Para referirse al resultado se puede usar \res en vez de escribir "resultado"
%
% EJEMPLO 
% \begin{interfaz}{encolar}{inout $c$: cola(int), in $elem$: int}{}{\Ode{1}}
%	 \pre{c = c_0}
%	 \post{c = encolar(c_0, e)}
% \end{interfaz}
% 
% 


%%%%%%%%%%%%%%%%%%%%%
%%% IMPLEMENTACI�N %%%
%%%%%%%%%%%%%%%%%%%%%

%%%Contadores usados%%%
\newcounter{linenumber}
\newcounter{showln}
\newcounter{tabs}
\newcounter{backtabs}

%%%Envireonment para interfaces%%%
\newenvironment{interfaz}[4]{

	% Primero definamos \pre y \post
	\newcommand{\pre}[1]{\ensuremath{\left\{\begin{array}{l}##1\end{array}\right\}}\par}
	\newcommand{\post}[1]{\pre{##1}}

	% Para comentarios
	\newcommand{\nota}[1]{\begin{small}\textbf{Nota:} ##1\end{small}}

	% El prefijo i de las funciones imperativas
	\renewcommand{\i}{\textit{i}}

	% Por si alg�n d�a dejamos de usar resultado.
	\newcommand{\res}{resultado}


  % Aridad de la funci�n
  \noindent
  	\textsl{#1}(#2) 	
  	\ifthenelse{\equal{#3}{}}{}{ $\rightarrow$ \textit{\res}: #3}  	
  	\ifthenelse{\equal{#4}{}}{}{\hspace{1 ex} #4}
  	
  % Reducimos un touch el indent
  \setlength{\parindent}{1 ex}
  \par
}{
	% Un espacio en blanco al final
	\vspace{3 ex}
} junto con los \usepackage
%
% EL ENVIRONMENT INTERFAZ
% Se usa con \begin{interfaz} y \end{interfaz}.
% Toma 4 par�metros:
%  - Nombre de la funci�n
%  - Par�metros
%  - Tipo del resultado (Opcional)
%  - Complejidad (Oppcional)
% 
% Para las funciones sobre estr se puede poner \i antes del nombre de la funci�n para una i m�s pro. Eg: \begin{interfaz}{\i Desencolar} etc.
% Para poner pre y postcondiciones se usan las macros \pre y \post (No hace falta poner math mode para usar s�mbolos)
% Se puede agregar un comentario usando \nota{el comentario}
% Para referirse al resultado se puede usar \res en vez de escribir "resultado"
%
% EJEMPLO 
% \begin{interfaz}{encolar}{inout $c$: cola(int), in $elem$: int}{}{\Ode{1}}
%	 \pre{c = c_0}
%	 \post{c = encolar(c_0, e)}
% \end{interfaz}
% 
% 


%%%%%%%%%%%%%%%%%%%%%
%%% IMPLEMENTACI�N %%%
%%%%%%%%%%%%%%%%%%%%%

%%%Contadores usados%%%
\newcounter{linenumber}
\newcounter{showln}
\newcounter{tabs}
\newcounter{backtabs}

%%%Envireonment para interfaces%%%
\newenvironment{interfaz}[4]{

	% Primero definamos \pre y \post
	\newcommand{\pre}[1]{\ensuremath{\left\{\begin{array}{l}##1\end{array}\right\}}\par}
	\newcommand{\post}[1]{\pre{##1}}

	% Para comentarios
	\newcommand{\nota}[1]{\begin{small}\textbf{Nota:} ##1\end{small}}

	% El prefijo i de las funciones imperativas
	\renewcommand{\i}{\textit{i}}

	% Por si alg�n d�a dejamos de usar resultado.
	\newcommand{\res}{resultado}


  % Aridad de la funci�n
  \noindent
  	\textsl{#1}(#2) 	
  	\ifthenelse{\equal{#3}{}}{}{ $\rightarrow$ \textit{\res}: #3}  	
  	\ifthenelse{\equal{#4}{}}{}{\hspace{1 ex} #4}
  	
  % Reducimos un touch el indent
  \setlength{\parindent}{1 ex}
  \par
}{
	% Un espacio en blanco al final
	\vspace{3 ex}
}