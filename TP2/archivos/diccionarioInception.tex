
\section{M\'odulo Diccionario Inception}

\subsection{Especificaci\'on}
	\begin{tad}{\nombretad{Diccionario Inception}}
	
	Extiende al TAD Diccionario Finito. \\
	TAD Diccionario Finito es: dicc(cliente, diccionarioDS(nombre, nat))\\
	Donde diccionarioDS es un renombre de diccionario.\\
	\\
	Generos: diccIncep\\
		
	Donde las siguientes operaciones son renombres de:
	
		\begin{tabular}[c]{|c|c|c|}
		\hline
		\textbf{DiccIncep} & \textbf{DiccFinito} \\
		\hline
		vacioDI() & Vacio() \\
		\hline
		Definido?DI(d,c) & Def?(c,d) \\
		\hline
		ClientesDI(d) & Claves(d) \\
		\hline
		ObtenerDI(d, c) & Obtener(c,d) \\
		\hline
		AgregarClienteDI(d, c) & Definir(c, crearDiccString(), d) \\
		\hline		
	\end{tabular}\\\\
	
	
EXPORTA:	
	
\otrasops
\func{agregarTitulo}{d/diccIncep, c/cliente, nT/nombre, n/nat}{dicc}{$\lbrace$c $\in$clientesDI(d)$\rbrace$}
\func{sumarAcciones}{d/diccIncep, c/cliente, nT/nombre, n/nat}{dicc}{$\lbrace$c $\in$ clientesDI(d) \yluego nT $\in$ titulosDe(d, c)$\rbrace$}
\func{restarAcciones}{d/diccIncep, c/cliente, nT/nombre, n/nat}{dicc}{$\lbrace$c $\leq$ cantidad(d, c, nT) $\wedge$ c $\in$ clientesDI(d) \yluego nT $\in$ titulosDe(d, c)$\rbrace$}
\func{titulosDe}{d/diccIncep, c/cliente}{conj(nombre)}{$\lbrace$c $\in$clientesDI(d)$\rbrace$}
\func{cantidad}{d/diccIncep, c/cliente, nT/nombre}{nat}{$\lbrace$c $\in$ clientesDI(d) \yluego nT $\in$ titulosDe(d, c)$\rbrace$}
\func{accionesTotales}{d/diccIncep, c/cliente}{nat}{$\lbrace$c $\in$ clientesDI(d)$\rbrace$}
\func{sumatoriaDeAcciones}{d/diccIncep, c/cliente, conj/conj(nombre)}{nat}{$\lbrace$c $\in$ clientesDI(d) \yluego nT $\in$ titulosDe(d, c)$\rbrace$}

\axiomas{\paratodo{diccIncep}{d}, \paratodo{cliente}{c}, \paratodo{nat}{n}, \paratodo{conj}{conj(nombre)}}
\alinearaxiomas{}{}

\axioma{agregarTitulo(d, c, nT, n)}{definirDS(obtener(d,c), nT, n)} 
\axioma{sumarAcciones(d, c, nT, n)}{definirDS(obtener(d,c), nT, cantidad(d, c, nT)+n)}
\axioma{restarAcciones(d, c, nT, n)}{definirDS(obtener(d,c), nT, cantidad(d, c, nT)-n)}
\axioma{titulosDe(d, c)}{clavesDS(obtener(d,c))}
\axioma{cantidad(d, c, nT)}{obtenerDS(obtener(d,c))}
\axioma{accionesTotales(d, c}{sumatoriadeAcciones(d, c, titulosDe(d, c))}
\axioma{sumatoriaDeAcciones(d, c, $\emptyset$)}{0}
\axioma{sumatoriaDeAcciones(d, c, Ag(nT, conj))}{cantidad(d, c, nT) + sumatoriaDeAcciones(d, c, conj)}

\end{tad}

\subsection{Servicios Exportados}
	\begin{tabular}[c]{|c|c|c|}
		\hline
		Operacion & Aliasing & Complejidad \\
		\hline
		vacioDI & No &  O(1)\\
		\hline
		AgregarClienteDI & No &  O(C)\\
		\hline
		Definido?DI & No &  O(log C)\\
		\hline
		ClientesDI & No &  O(C)\\
		\hline
		obtenerDI & Si &  O(log C)\\
		\hline
		AgregarTitulo & No &  O(C + |nT|)\\
		\hline
		SumarAcciones & Si &  O(log C + |nT|)\\
		\hline
		RestarAcciones & Si &  O(log C + |nT|)\\
		\hline
		TitulosDe & Si &  O(C)\\
		\hline
		Cantidad & Si &  O(log C + |nT|)\\
		\hline
		AccionesTotales & Si &  O(log C)\\
		\hline
	\end{tabular}\\\\

	
\subsection{Interfaz}
	Interfaz \nombretad{Diccionario Inception}\\
	Se explica con: TAD Diccionario Inception \\
	Que es dicc(cliente, diccionarioDS(nombre, nat))\\
	Donde TAD dicc es Diccionario Finito y diccionarioDS es un renombre de diccionario.\\
	Se representa extendiendo: dicc(cliente, diccionarioDS(nombre, nat))\\
	Generos: dIncep\\
	Operaciones:\\
	
	\begin{interfaz}{VacioDI}{}{dIncep}{}
	\textbf{Pre} $\equiv$ \pre{true}
	\textbf{Post} $\equiv$ \post{resultado \igobs vacio}
	\end{interfaz}
	
	\begin{interfaz}{AgregarClienteDI}{\param{inout}{dicc}{dIncep}, \param{in}{clave}{cliente}}{dIncep}{}
	\textbf{Pre} $\equiv$ \pre{dicc \igobs dicc_0 \wedge clave \notin claves(dicc_0)}
	\textbf{Post} $\equiv$ \post{resultado \igobs AgregarClienteDI(dicc_0, clave)}
	\end{interfaz}
	
	\begin{interfaz}{Definido?DI}{\param{in}{dicc}{dIncep}, \param{in}{clave}{cliente}}{bool}{}
	\textbf{Pre} $\equiv$ \pre{dicc \igobs dicc_0}
	\textbf{Post}$\equiv$ \post{resultado \igobs definido?DI(clave, dicc_0)}
	\end{interfaz}
		
	\begin{interfaz}{ClientesDI}{\param{in}{dicc}{dIncep}}{conj(cliente)}{}
	\textbf{Pre} $\equiv$ \pre{dicc \igobs dicc_0}
	\textbf{Post} $\equiv$ \post{resultado \igobs ClientesDI(dicc_0)}
	\end{interfaz}		
		
		\begin{interfaz}{ObtenerDI}{\param{in}{dicc}{dIncep}, \param{in}{clave}{cliente}}{diccionarioDS(nombre, nat)}{}
	\textbf{Pre} $\equiv$ \pre{dicc \igobs dicc_0 \wedge def?(clave, dicc) \equiv true}
	\textbf{Post} $\equiv$ \post{resultado \igobs obtenerDI(dicc_0, clave)}
	\end{interfaz}

		\begin{interfaz}{AgregarTitulo}{\param{inout}{dicc}{dIncep}, \param{in}{clave}{cliente}, \param{in}{nT}{nombre}, \param{in}{c}{nat}}{dIncep}{}
	\textbf{Pre} $\equiv$ \pre{dicc \igobs dicc_0 \wedge def?(nT, obtener(clave, dicc).diccionario) \equiv false}
	\textbf{Post} $\equiv$ \post{resultado \igobs agregarTitulo(dicc_0, clave, nT, c)}
	\end{interfaz}	

	\begin{interfaz}{SumarAcciones}{\param{inout}{dicc}{dIncep}, \param{in}{clave}{cliente}, \param{in}{nT}{nombre}, \param{in}{c}{nat}}{dIncep}{}
	\textbf{Pre} $\equiv$ \pre{dicc \igobs dicc_0  \wedge 
	\endline def?(clave, dicc) \equiv true  \wedge
	\endline (nT \in claves(obtener(clave, dicc).diccionario \equiv true)}
	\textbf{Post} $\equiv$ \post{resultado \igobs sumarAcciones(dicc_0, clave, nT, c)}
	\end{interfaz}
	
	\begin{interfaz}{RestarAcciones}{\param{inout}{dicc}{dIncep}, \param{in}{clave}{cliente}, \param{in}{nT}{nombre}, \param{in}{c}{nat}}{dIncep}{}
	\textbf{Pre} $\equiv$ \pre{dicc \igobs dicc_0  \wedge 
	\endline def?(clave, dicc) \equiv true  \wedge
	\endline (nT \in claves(obtener(clave, dicc) \equiv true)}
	\textbf{Post} $\equiv$ \post{resultado \igobs restarAcciones(dicc_0, clave, nT, c)}
	\end{interfaz}
	
	\begin{interfaz}{TitulosDe}{\param{in}{dicc}{dIncep}, \param{in}{clave}{cliente}}{conj(Nombre)}{}
	\textbf{Pre} $\equiv$ \pre{dicc \igobs dicc_0 \wedge definido?DI(dicc, clave)}
	\textbf{Post} $\equiv$ \post{resultado \igobs TitulosDe(dicc_0, clave)}
	\end{interfaz}	
	
	\begin{interfaz}{Cantidad}{\param{in}{dicc}{dIncep}, \param{in}{clave}{cliente}, \param{in}{nT}{nombre}}{nat}{}
	\textbf{Pre} $\equiv$ \pre{dicc \igobs dicc_0}
	\textbf{Post} $\equiv$ \post{resultado \igobs Cantidad(dicc_0, clave, nT)}
	\end{interfaz}
	
		\begin{interfaz}{accionesTotales}{\param{in}{e}{estr}, \param{in}{c}{cliente}}{nat}{}
	\textbf{Pre} $\equiv$ \pre{dicc \igobs dicc_0}
	\textbf{Post} $\equiv$ \post{resultado \igobs accionesTotales(dicc_0, c)}
	\end{interfaz}
	
	
	
\subsection{Estructura de Representaci\'on}
	diccIncep($\alpha$) se representa con estr$_-$diccIncep\\
	Donde estr$_-$diccIncep es dicc\_Finito(cliente, tupla<
													titulos:	diccionarioDS(nombre, nat),
													accionesTotales:	nat
													>)\\
	
	
\subsection{Invariante de Representaci\'on}

Vale el invariante del Tad Diccionario Finito, y ademas: \\
$\bullet$ Las Acciones Totales de cada cliente es la suma de las acciones de cada titulo. \\


\noindent Rep: $\widehat{diccIncep}$ \en bool\\
	($\forall$ e: $\widehat{diccIncep}$) Rep(dicc) = $ true$ $\Leftrightarrow$ \\
	Rep(DiccionarioFinito) $\wedge$ \\	
	($\forall$ c: cliente) definido?DI(e, c) \impluego 
	($\forall$ nT: nombre) nT $\in$ TitulosDe(e, c) \impluego \\
	ObtenerDS(Obtener(e, c), nT).accionesTotales == sumatoriaDeAcciones(e, c, titulosDe(e, c)) \\
	



\subsection{Funci\'on de Abstracci\'on}

Vale la funcion de Abstraccion del Modulo Diccionario Finito. Ademas\\


	\noindent Abs: $\widehat{estr}$ e \en diccIncep (Rep(e))\\
	\begin{tabular}[t]{@{} l @{} @{} l @{}}
	($\forall$ e: $\widehat{estr}$) Abs(e) $\equiv$ d: $\widehat{diccIncep}$ /  \\
	$\left(
	\begin{array}{l}
	(( \forall c:cliente) c \in clientesDI(d) \impluego \\
	titulosDe(d, c) \igobs titulosDe(e, c) \wedge accionesTotales(d, c)==accionesTotales(e, c) \wedge \\
	( \forall nT:nombre, \forall n:nat)agregarTitulo(d, c, nT, n) \igobs agregarTitulo(e, c, nT, n) \wedge \\
	( \forall nT:nombre) nT \in clavesDS(obtener(c, d)) \impluego \\
		sumarAcciones(d, c, nT, c) \igobs sumarAcciones(e, c, nT, c) \wedge \\		
		restarAcciones(d, c, nT, c) \igobs restarAcciones(e, c, nT, c) \wedge \\
		cantidad(d, c, nT) = cantidad(e, c, nT)
	\end{array} 
	\right)$\\
	\end{tabular}\\\\



\subsection{Algoritmos}
	En la Complejidad: C es la cantidad de clientes en el Wolfie, nT el nombre del titulo \\
	\\
\begin{algorithm}{$i$VacioDI}{}{\param{}{resultado}{estr}}
				\RETURN Vacio()
			\end{algorithm}
	\textit{Complejidad:} \textbf{O(1)} \textsubscript{Vacio()}\\
	\textit{Aliasing:} No \\

\begin{algorithm}{$i$AgregarClienteDI}{\param{inout}{e}{estr}, \param{in}{c}{cliente}}{}
				definir(c, vacioDS(),e) 			\\	
				obtener(e, c).accionesTotales \leftarrow 0\\
			\end{algorithm}
	\textit{Complejidad:} O(C*copiarInfo)\textsubscript{definir} + O(log C) \textsubscript{obtener}\\
	 Donde O(copiarInfo)=O(vacioDS)=O(1)\\
	$\Rightarrow$ O(C * 1 + log C)=\textbf{O(C)}\\
	\textit{Aliasing:} No \\
				
\begin{algorithm}{$i$Definido?DI}{\param{in}{e}{estr}, \param{in}{c}{cliente}}{\param{}{resultado}{bool}}
				\RETURN definida?(c, e)
			\end{algorithm}	
	\textit{Complejidad:} \textbf{O(log C)} \textsubscript{definida?}\\		
	\textit{Aliasing:} No \\
			
\begin{algorithm}{$i$ClientesDI}{\param{in}{e}{estr}}{\param{}{resultado}{conj(cliente)}}
				\RETURN claves(e)		
			\end{algorithm}	
			\textit{Complejidad:} \textbf{\textbf{O(C)}} \textsubscript{claves}\\	
	\textit{Aliasing:} No \\	
			
\begin{algorithm}{$i$ObtenerDI}{\param{in}{e}{estr}, \param{in}{c}{cliente}}{\param{}{resultado}{diccionarioDS(nombre, nat)}}
				\RETURN obtener(c, e)	
			\end{algorithm}
			\textit{Complejidad:} \textbf{O(log C)} \textsubscript{obtener}\\	
	\textit{Aliasing:} Si \\	
			
\begin{algorithm}{$i$agregarTitulo}{\param{inout}{e}{estr}, \param{in}{c}{cliente}, \param{in}{nT}{nombre}, \param{in}{n}{nat}}{}
				\VAR{punt\!: punteroAdiccionarioDS} \\
				punt \leftarrow \&obtener(e, c)\\
				(*punt).definirDS(nT, n)
			\end{algorithm}
			\textit{Complejidad:} O(log C)\textsubscript{obtener}+O(|nT|)\textsubscript{definirDS} = \textbf{O(C + |nT|)}\\
	\textit{Aliasing:} No \\
			\\
						
\begin{algorithm}{$i$SumarAcciones}{\param{inout}{e}{estr}, \param{in}{c}{cliente}, \param{in}{nT}{nombre}, \param{in}{n}{nat}}{}
				\VAR{tenia\!: nat} \\
				tenia \leftarrow cantidad(e, c, nT)\\
				\VAR{punt\!: punteroAdiccionarioDS} \\
				punt \leftarrow \&obtener(e, c)\\
				(*punt).definirDS(nT, tenia + n)\\
				obtener(e, c).accionesTotales \leftarrow obtener(e, c).accionesTotales + n	
			\end{algorithm}	
			\textit{Complejidad:} O(|nT| + log C) \textsubscript{cantidad} + O(log C)\textsubscript{obtener}+O(|nT|)\textsubscript{definirDS} + \\ O(log C) \textsubscript{obtener} + O(log C) \textsubscript{obtener}=\\
			O(|nT| + log C + log C + |nT| + log C + log C) =\textbf{ O(|nT| + log C)}\\
	\textit{Aliasing:} Si \\
								

\begin{algorithm}{$i$RestarAcciones}{\param{inout}{e}{estr}, \param{in}{c}{cliente}, \param{in}{nT}{nombre}, \param{in}{n}{nat}}{}
				\VAR{tenia\!: nat} \\
				tenia \leftarrow cantidad(e, c, nT)\\
				\VAR{punt\!: punteroAdiccionarioDS} \\
				punt \leftarrow \&obtener(e, c)\\
				(*punt).definirDS(nT, tenia - n)\\
				obtener(e, c).accionesTotales \leftarrow obtener(e, c).accionesTotales - n	
			\end{algorithm}	
			\textit{Complejidad:} O(|nT| + log C) \textsubscript{cantidad} 
			+ O(log C)\textsubscript{obtener}
			+O(|nT|)\textsubscript{definirDS}
			+ \\ O(log C) \textsubscript{obtener} 
			+ O(log C) \textsubscript{obtener}=\\
			O(|nT| + log C + log C + |nT| + log C + log C) =\textbf{ O(|nT| + log C)}\\
	\textit{Aliasing:} Si \\
			
\begin{algorithm}{$i$titulosDe}{\param{in}{e}{estr}, \param{in}{c}{cliente}}{\param{}{resultado}{conj(nombre)}}
			\RETURN clavesDS(obtener(d, c))
			\end{algorithm}	
			\textit{Complejidad:} O(C)\textsubscript{clavesDS} + O(log C)\textsubscript{obtener} =\\
			O(C + log C) = \textbf{O(C)} \\
	\textit{Aliasing:} Si \\
			
\begin{algorithm}{$i$cantidad}{\param{in}{e}{estr}, \param{in}{c}{cliente}, \param{in}{nT}{nombre}}{\param{}{resultado}{nat}}
			\VAR{cant\!: nat} \\
			\begin{IF}{definido?DS(obtener(e, c), nT)} \\
			cant \leftarrow obtenerDS(obtener(e, c), nT)
		\ELSE \\
			cant \leftarrow 0
		\end{IF} \\
		\RETURN cant
			\end{algorithm}	
			\textit{Complejidad:} O(|nT|) \textsubscript{definido?DS}  + O(log C) \textsubscript{obtener} +
			O(|nT|)\textsubscript{obtenerDS}  + O(log C) \textsubscript{obtener}= \\ \textbf{O(|nT| + log C)}\\
	\textit{Aliasing:} Si \\

\begin{algorithm}{$i$accionesTotales}{\param{in}{e}{estr}, \param{in}{c}{cliente}}{\param{}{resultado}{nat}}
		\RETURN obtener(e, c).accionesTotales
	\end{algorithm}	
		\textit{Complejidad:} \textbf{O(log C)} \textsubscript{obtener} \\		
	\textit{Aliasing:} Si \\

	
\subsection{Servicios usados}

 {\LARGE Diccionario Finito}\\

\begin{tabular}[c]{|l|l|l|}
		\hline
		Operacion & Aliasing & Complejidad \\
		\hline
		Vacio & No & O(n) \\
		\hline
		Definir & No & O(n * copiar info)\\
		\hline
		Definida? & No & O($\log$ n)\\
		\hline
		Obtener & Si & O($\log$ n)\\
		\hline
		Tama\~no & No & O(1)\\
		\hline
		Tama\~noMax & No & O(1)\\
		\hline
		Claves & No & O(n)\\
		\hline
	
	\end{tabular}\\\\
n es la cantidad de claves\\

 {\LARGE Diccionario String}\\
 \begin{tabular}[c]{|l|l|l|}
		\hline
		Operacion & Aliasing & Complejidad \\
		\hline
		CrearDiccString & No & O(1) \\
		\hline
		DefinirDS & No & O(|c|)\\
		\hline
		Definido?DS & No & O(|c|)\\
		\hline
		ObtenerDS & Si & O(|c|)\\
		\hline
		ClavesDS & Si & O(N) \\
		\hline
		CopiarDS & No & O(N*(|c\textsubscript{max}|+O(copiarSignificado))) \\
		\hline
		
		
		
	\end{tabular}\\\\

N es la cantidad de claves.\\
c es la clave, |c| es su largo.\\
\\
Usa el Modulo{\LARGE  Conjunto Lineal} de La Catedra. \\


	
\newpage