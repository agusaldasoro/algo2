\section{M\'odulo Informacion de Promesa}

\subsection{Especificaci\'on}
	\begin{tad}{\nombretad{Informacion de Promesa}}
	
	\igualobs{infP1}{infP2}{infoProm}{
	cliente(infP1)= cliente(infP2) \wedge \newline promesa(infP1)\igobs promesa(infP2) \wedge \newline
	accionesDelCliente(infP1)= accionesDelCliente(infP2)
	}


\generos{infoProm}
\exporta{generadores y observadores}
\usa{\nombretad{Promesa, Cliente, Nat}}


\observadores 
\func{cliente}{infoProm}{cliente}{}
\func{promesa}{infoProm}{promesa}{}
\func{accionesDelCliente}{infoProm}{nat}{}

\generadores
\func{nuevoinfoProm}{cliente/c, promesa/p, nat/n}{infoProm}{}

\axiomas{\paratodo{cliente}{c}, \paratodo{promesa}{p}, \paratodo{nat}{n}}
\alinearaxiomas{}{}

\axioma{cliente(nuevoinfoProm(c, p, n))}{c}
\axioma{promesa(nuevoinfoProm(c, p, n))}{p}
\axioma{accionesDelCliente(nuevoinfoProm(c, p, n))}{n}

\end{tad}



\subsection{Servicios Exportados}
\begin{tabular}[c]{|l|l|l|}
		\hline
		Operacion & Aliasing & Complejidad \\
		\hline
		nuevoinfoProm & No & O(|nombre(t)|) \\
		\hline
		cliente & No & O(1) \\
		\hline
		promesa & No & O(1) \\
		\hline
		acccionesDelCliente & No & O(1) \\
		\hline
	\end{tabular}\\\\


\subsection{Interfaz}
	\noindent Interfaz \nombretad{Informacion de Promesa}\\
	Se explica con: TAD Informacion de Promesa\\
	G\'{e}neros: infoProm\\
	
	Operaciones:\\
	
	\begin{interfaz}{nuevoinfoProm}{\param{in}{c}{cliente}, \param{in}{p}{promesa}, \param{in}{n}{nat}}{infoProm}{}
	\textbf{Pre} $\equiv$\pre{true}
	\textbf{Post} $\equiv$\post{resultado \igobs nuevoinfoProm(c, p, n)}
	\end{interfaz}
	
	\begin{interfaz}{cliente}{\param{in}{iP}{infoProm}}{cliente}{}
	\textbf{Pre} $\equiv$\pre{true}
	\textbf{Post} $\equiv$\post{resultado = cliente(iP)}
	\end{interfaz}
	
	\begin{interfaz}{promesa}{\param{in}{iP}{infoProm}}{promesa}{}
	\textbf{Pre} $\equiv$\pre{true}
	\textbf{Post} $\equiv$\post{resultado \igobs promesa(iP)}
	\end{interfaz}
	
	\begin{interfaz}{accionesDelCliente}{\param{in}{iP}{infoProm}}{nat}{}
	\textbf{Pre} $\equiv$\pre{true}
	\textbf{Post} $\equiv$\post{resultado = accionesDelCliente(iP)}
	\end{interfaz}
	

\subsection{Estructura de Representaci\'on}

Estr es tupla<cliente: cliente, promesa: promesa, accionesDelCliente: punteroANat>


\subsection{Invariante de Representacii\'on}

	\noindent Rep: $\widehat{estr}$ \en bool\\
	\begin{tabular}[t]{@{} r @{} @{} l @{}}
	($\forall$ e: $\widehat{estr}$) Rep(e) $\Leftrightarrow$&
	$				
	\left(
	\begin{array}{l}
	true
	\end{array} 
	\right)$\\
	\end{tabular}\\\\
\subsection{Funci\'on de Abstracci\'on}

	\noindent Abs: $\widehat{infoProm}$ e \en $\widehat{infoProm}$ (Rep(e))\\
	\begin{tabular}[t]{@{} r @{} @{} l @{}}
	($\forall$ iP: $\widehat{infoProm}$)
 \! Abs(iP) $\equiv$ iprom \!: $\widehat{infoProm}$ $\Leftrightarrow$&
	$				
	\left(
	\begin{array}{l}
		e.cliente = cliente(iprom) \wedge \\ e.promesa=_{obs}promesa(iprom) \wedge \\ (*e.accionesDelCliente)=accionesDelCliente(iprom)
	\end{array} 
	\right)$\\
	\end{tabular}\\\\

\subsection{Algoritmos}
	\begin{algorithm}{$i$nuevoinfoProm}{\param{in}{c}{cliente}, \param{in}{p}{promesa}, \param{in}{pN}{punteroANat}}{\param{}{resultado}{infoProm}}
	\VAR{infoP\!: infoProm} \\
	infoP.cliente \leftarrow c\\
	infoP.promesa \leftarrow copiar(p)\\
	infoP.accionesDelCliente \leftarrow pN\\
	\RETURN infoT
	\end{algorithm}

	\begin{algorithm}{$i$cliente}{\param{in}{e}{estr}}{\param{}{resultado}{cliente}}
		\RETURN e.cliente
	\end{algorithm}

	
	\begin{algorithm}{$i$promesa}{\param{in}{e}{estr}}{\param{}{resultado}{promesa}}
		\RETURN e.promesa
	\end{algorithm}

	
	\begin{algorithm}{$i$accionesDelCliente}{\param{in}{e}{estr}}{\param{}{resultado}{nat}}
		\RETURN (*e.accionesDelCliente)
	\end{algorithm}

\subsection{Servicios usados}


{\LARGE Promesa} \\

\begin{tabular}[c]{|c|c|c|}
	
	\hline
		Operacion & Aliasing & Complejidad \\
		\hline
		T\'itulo & Si &  O(1)\\
		\hline
		Tipo & No & O(1)\\
		\hline
		L\'imite & No & O(1)\\
		\hline
		Cantidad& No & O(1)\\
		\hline
		CrearPromesa & No & O(1)\\
		\hline
		CopiarPromesa & No & O(|nombre(t)|)\\
		\hline
	\end{tabular}\\\\
\newpage