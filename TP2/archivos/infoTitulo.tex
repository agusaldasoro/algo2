\section{M\'odulo Informacion de Titulo}

\subsection{Especificaci\'on}
	\begin{tad}{\nombretad{Informacion de Titulo}}
	
	\igualobs{infT1}{infT2}{infoTitulo}{
	titulo(infT1)\igobs titulo(infT2) \wedge \newline
	ventas(infT1)\igobs ventas(infT2) \wedge \newline
	compras(infT1)\igobs compras(infT2) \wedge \newline 
	disponibles(infT1)\igobs disponibles(infT2)	
	}


\generos{infoTitulo}
\exporta{generadores y observadores}
\usa{\nombretad{Promesa, Cliente, Nat, Conj(infoProm), Arreglo\_Ordenable(infoProm)}}


\observadores 
\func{titulo}{infoTitulo}{titulo}{}
\func{Ventas}{infoTitulo}{conj(infoProm)}{}
\func{Compras}{infoTitulo}{arreglo\_ordenable(infoProm)}{}
\func{disponibles}{infoTitulo}{nat}{}

\generadores
\func{nuevoinfoTitulo}{titulo/t,conj(infoProm)/v,arreglo\textunderscore ordenable(infoProm)/c, nat/n}{infoTit}{}

\axiomas{\paratodo{titulo}{t}, \paratodo{infoProm}{v, c}, \paratodo{nat}{d}}
\alinearaxiomas{}{}

\axioma{titulo(nuevoInfoTitulo(t, v, c, d))}{t}
\axioma{ventas(nuevoInfoTitulo(t, v, c, d))}{v}
\axioma{compras(nuevoInfoTitulo(t, v, c, d))}{c}
\axioma{disponibles(nuevoInfoTitulo(t, v, c, d))}{d}

\end{tad}

\subsection{Servicios Exportados}
\begin{tabular}[c]{|l|l|l|}
		\hline
		Operacion & Aliasing & Complejidad \\
		\hline
		nuevoInfoTitulo & No & O(copy(V)+copy(C)) \\
		\hline
		titulo & Si & O(1) \\
		\hline
		ventas & Si & O(1) \\
		\hline
		compras & Si & O(1) \\
		\hline
		disponibles & Si & O(1) \\
		\hline
	\end{tabular}\\\\
	Donde V es un conjunto de infoProm y C un arreglo ordenable de infoProm

\subsection{Interfaz}
	\noindent Interfaz \nombretad{Informacion de Titulo}\\
	Se explica con: TAD Informacion de Titulo\\
	G\'{e}neros: infoTit\\
	Operaciones:
	
	
	\begin{interfaz}{nuevoInfoTitulo}{\param{in}{t}{titulo}, \param{in}{v}{infoProm}, \param{in}{c}{infoProm}, \param{in}{d}{nat}}{infoTit}{}
	\textbf{Pre} $\equiv$\pre{true}
	\textbf{Post} $\equiv$\post{resultado \igobs nuevoInfoTitulo(t, v, c, d)}
	\end{interfaz}
	
	\begin{interfaz}{titulo}{\param{in}{iT}{infoTit}}{titulo}{}
	\textbf{Pre} $\equiv$\pre{true}
	\textbf{Post} $\equiv$\post{resultado \igobs titulo(iT)}
	\end{interfaz}
	
	\begin{interfaz}{ventas}{\param{in}{iT}{infoTit}}{conj(infoProm)}{}
	\textbf{Pre} $\equiv$\pre{true}
	\textbf{Post} $\equiv$\post{resultado \igobs ventas(iT)}
	\end{interfaz}
	
	\begin{interfaz}{compras}{\param{in}{iT}{infoTit}}{conj(infoProm)}{}
	\textbf{Pre} $\equiv$\pre{true}
	\textbf{Post} $\equiv$\post{resultado \igobs compras(iT)}
	\end{interfaz}
	
	\begin{interfaz}{disponibles}{\param{in}{iT}{infoTit}}{nat}{}
	\textbf{Pre} $\equiv$\pre{true}
	\textbf{Post} $\equiv$\post{resultado \igobs disponibles(iT)}
	\end{interfaz}
	
\subsection{Estructura de Representaci\'on}

estr es	tupla< \\
			elIterador: itConj(titulo), \\
			Ventas:		conj(infoProm),  \\
			Compras:	arreglo\_ordenable(infoProm)  \\
			Disponibles: nat \\
			> \\

\subsection{Invariante de Representaci\'on}

1)Todas las promesas son del titulo al cual apunta el iterador \\
2)No existe un cliente que tenga dos promesas en ventas\\
3)No existe un cliente que tenga dos promesas en compras \\
4)Ventas solo tiene promesas de ventas \\
5)Compras solo tiene promesas de compras \\


\noindent Rep: $\widehat{estr}$ \en bool\\
	\begin{tabular}[t]{@{} r @{} @{} l @{}}
	($\forall$ e: $\widehat{estr}$) Rep(e) $\Leftrightarrow$&
					
	$\left(
	\begin{array}{l}
	((\forall pr: infoProm) pr \in damePromesas(e.compras) \impluego \\
	((tipo(promesa(pr))==compra) \wedge (titulo(promesa(pr))==nombre(siguiente(e.elIterador))))) \wedge \\
	((\forall pr: infoProm) pr \in e.ventas \impluego \\
	((tipo(promesa(pr))==venta) \wedge (titulo(promesa(pr))==nombre(siguiente(e.elIterador))))) \wedge \\
	((\forall pr1, pr2: infoProm) pr1, pr2 \in damePromesas(e.compras) \Rightarrow \\
	(cliente(pr1)==cliente(pr2) \Leftrightarrow pr1\igobs pr2)) \\
	((\forall pr1, pr2: infoProm) pr1, pr2 \in e.ventas \Rightarrow \\
	(cliente(pr1)==cliente(pr2) \Leftrightarrow pr1\igobs pr2))
	\end{array} 
	\right)$\\
	\end{tabular}\\\\

\subsection{Funci\'on de Abstracci\'on}

	\noindent Abs: $\widehat{infoTit}$ e \en $\widehat{infoTitulo}$ (Rep(e))\\
	\begin{tabular}[t]{@{} r @{} @{} l @{}}
	($\forall$ iT: $\widehat{infoTit}$)
 \! Abs(iT) $\equiv$ it \!: $\widehat{infoTit}$ $\Leftrightarrow$&
	$				
	\left(
	\begin{array}{l}
		titulo(iT)\igobs (siguiente(e.elIterador)) \wedge \\
		Ventas(iT)\igobs e.Ventas \wedge \\
		Compras(iT) \igobs e.Compras \wedge \\
		Disponibles(iT) \igobs e.Disponibles 
	\end{array} 
	\right)$\\
	\end{tabular}\\\\

\subsection{Algoritmos}

\begin{algorithm}{$i$nuevoInfoTitulo}{\param{in}{pT}{itConj(titulo)}, \param{in}{v}{conj(infoProm)}, \param{in}{c}{arreglo\_ordenable(infoProm)}, \param{in}{d}{nat}}{\param{}{resultado}{infoTit}}
	\VAR{infoT\!: infoTitulo} \\
	infoT.titulo \leftarrow pT\\
	infoT.ventas \leftarrow copiar(v)\\
	infoT.compras \leftarrow copiar(c)\\
	infoT.disponibles \leftarrow d\\
	\RETURN infoT
	\end{algorithm}

	
	\begin{algorithm}{$i$titulo}{\param{in}{e}{estr}}{\param{}{resultado}{titulo}}
		\RETURN siguiente(e.elIterador)
	\end{algorithm}


	\begin{algorithm}{$i$ventas}{\param{in}{e}{estr}}{\param{}{resultado}{conj(infoProm)}}
		\RETURN e.ventas
	\end{algorithm}

	
	\begin{algorithm}{$i$compras}{\param{in}{e}{estr}}{\param{}{resultado}{arreglo\_ordenable(infoProm)}}
		\RETURN e.compras
	\end{algorithm}

	
	\begin{algorithm}{$i$disponibles}{\param{in}{e}{estr}}{\param{}{resultado}{nat}}
		\RETURN e.disponibles
	\end{algorithm}


\subsection{Servicios usados}

{\Large Arreglo\_ordenable}\\

\begin{tabular}[c]{|l|l|l|}
		\hline
		Operacion & Aliasing & Complejidad \\
		\hline
		Crear & No & O(n) \\
		\hline
		DameIndice & Si & O(1)\\
		\hline
		Ordenar & No & O(n * $\log$ n)\\
		\hline
		Insertar & No & O(1)\\
		\hline
		Borrar & No & O(1)\\
		\hline
		Tama\~no & No & O(1)\\
		\hline
		DamePromesas & No & O(1)\\
		\hline
	
	\end{tabular}\\\\
n es el largo del arreglo\\
\\
{\LARGE Informacion de Promesa }
\\
\begin{tabular}[c]{|l|l|l|}
		\hline
		Operacion & Aliasing & Complejidad \\
		\hline
		nuevoinfoProm & No & O(|nombre(t)|) \\
		\hline
		cliente & No & O(1) \\
		\hline
		promesa & No & O(1) \\
		\hline
		acccionesDelCliente & No & O(1) \\
		\hline
	\end{tabular}\\\\


{\LARGE Promesa} \\

\begin{tabular}[c]{|c|c|c|}
	
	\hline
		Operacion & Aliasing & Complejidad \\
		\hline
		T\'itulo & Si &  O(1)\\
		\hline
		Tipo & No & O(1)\\
		\hline
		L\'imite & No & O(1)\\
		\hline
		Cantidad& No & O(1)\\
		\hline
		CrearPromesa & No & O(1)\\
		\hline
		CopiarPromesa & No & O(|nombre(t)|)\\
		\hline
	\end{tabular}\\\\
	
{\LARGE Titulo} \\

\begin{tabular}[c]{|c|c|c|c|}
	
	\hline
		Operacion & Aliasing & Complejidad \\
		\hline
		Nombre & Si & O(1)  \\
		\hline
		\#maxAcciones & No & O(1) \\
		\hline
		Cotizaci\'on & No & O(1)\\
		\hline
		EnAlza & No & O(1)\\
		\hline
		CrearT\'itulo & No & O(1)\\
		\hline
		Recotizar & No & O(1)\\
		\hline
		CopiarT\'itulo & No & O(|nombre(t)|)\\
		\hline
	\end{tabular}\\\\
	\\
Usa el Modulo {\LARGE Conjunto Lineal} de la Catedra \\
\newpage