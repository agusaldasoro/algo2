\section{M�dulo Diccionario String}

\subsection{Servicios Exportados}

\begin{tabular}[c]{|l|l|l|}
		\hline
		Operacion & Aliasing & Complejidad \\
		\hline
		CrearDiccString & No & O(1) \\
		\hline
		DefinirDS & No & O(|c|)\\
		\hline
		Definido?DS & No & O(|c|)\\
		\hline
		ObtenerDS & Si & O(|c|)\\
		\hline
		ClavesDS & Si & O(N) \\
		\hline
		CopiarDS & No & O(N*(|c\textsubscript{max}|+O(copiarSignificado))) \\
		\hline
		
		
		
	\end{tabular}\\\\

N es la cantidad de claves.\\
c es la clave, |c| es su largo.

Asumimos:\\
Todos los arreglos empiezan en 1\\
Para los strings asumimos que a cada char se puede acceder en O(1)\\
y que se puede obtener la longitud en O(N) y su creacion es proporcional a la longitud.\\
Asumimos que la posicion i del array corresponde a la letra i del abecedario.\\
Longitud: long(string)\\
Acceso: String[nat]\\
int(char) nos da el numero entre 1 y 27 que le corresponde en el abecedario. \\
\textbf{TAD Clave} es String


\subsection{Interfaz}
	Interfaz \nombretad{Diccionario String(Clave, Significado)}\\
	Se explica con: Diccionario(Clave, Significado)\\
	G\'{e}neros: diccionarioDS\\
	Operaciones:\\
	
	\begin{interfaz}{CrearDiccString}{}{diccionarioDS}{}
	\textbf{Pre} $\equiv$\pre{true}
	\textbf{Post} $\equiv$\post{resultado = vacio}
	\end{interfaz}
	
	\begin{interfaz}{DefinirDS}{\param{inout}{d}{diccionarioDS}, \param{in}{c}{clave}, \param{in}{s}{significado}}{}{}
	\textbf{Pre} $\equiv$\pre{d=d_0}
	\textbf{Post} $\equiv$\post{d= definir(c,s,d_0)}
	\end{interfaz}
	
	\begin{interfaz}{Definido?DS}{\param{inout}{d}{diccionarioDS}, \param{in}{c}{clave}}{bool}{}
	\textbf{Pre} $\equiv$\pre{true}
	\textbf{Post} $\equiv$\post{resultado = def?(c,d)}
	\end{interfaz}
	
	\begin{interfaz}{ObtenerDS}{\param{inout}{d}{diccionarioDS}, \param{in}{c}{clave}}{significado}{}
	\textbf{Pre} $\equiv$\pre{def?(c,d)}
	\textbf{Post} $\equiv$\post{resultado = obtener(c,d)}
	\end{interfaz}
	
	\begin{interfaz}{ClavesDS}{\param{inout}{d}{diccionarioDS}}{conj(clave)}{}
	\textbf{Pre} $\equiv$\pre{true}
	\textbf{Post} $\equiv$\post{resultado = claves(d)}
	\end{interfaz}
	
	\begin{interfaz}{CopiarDS}{\param{inout}{d}{diccionarioDS}}{diccionarioDS}{}
	\textbf{Pre} $\equiv$\pre{true}
	\textbf{Post} $\equiv$\post{resultado = d}
	\end{interfaz}	
	

\subsection{Estructura de Representaci�n}
	diccionario se representa con estr\\
	Donde estr es tupla $<$ definiciones: nodo , claves: conj(string), conjSignificados: conj(significado) $>$\\
	Donde nodo es tupla $<$siguientes: arreglo[27] de puntero a nodo, significado: puntero a significado  $>$
	
	
	
\subsection{Invariante de Representaci�n}
	\noindent Rep: $\widehat{estr}$ \en bool\\
	\begin{tabular}[t]{@{} r @{} @{} l @{}}
	($\forall$ d: $\widehat{estr}$) Rep(d) $\Leftrightarrow$&
	$				
	\left(
	\begin{array}{l}
	$No hay ciclos en d.definiciones$\\
	$Todos los caminos con significados validos estan en claves$\\
	$Para todo elemento del conjunto claves existe un camino de nodos valido $\\$ que tiene un significado valido$\\
	$No hay caminos finales (con todos los siguientes NULL) que no tengan significado valido$\\
	$Todos los significados validos apuntan a algun elemento de e.conjSignificados$ \\$
	y todos los elementos de e.conjSignificados tienen algun significado valido $\\$
	 con un puntero hacia el.$ 
	\end{array} 
	\right)$\\
	\end{tabular}\\

\subsection{Funci�n de Abstracci�n}
	\noindent Abs: $\widehat{estr}$ e \en diccionario (Rep(d))\\
	\begin{tabular}[t]{@{} l @{} @{} l @{}}
	($\forall$ e: $\widehat{estr}$) Abs(e) $\equiv$ d: $\widehat{diccionario}$ /  \\
	$\left(
	\begin{array}{l}
	(\forall c: clave) def?(c,d) \Leftrightarrow $(la clave tiene su camino valido con significado valido) $  \yluego \\ 
	(\forall c: clave) def?(c,d) \impluego obtener(c,d) \igobs $(el significado en e.conjSignificados al cual apunta $\\$ el puntero del camino de c)$ \\
	\end{array} 
	\right)$\\
	\end{tabular}\\\\$
$
\subsection{Algoritmos}
	
	\begin{algorithm}{$i$CrearDiccString}{}{\param{}{resultado}{estr}}
	\VAR{i:nat}\\
	resultado.claves = Vacio()\\
	\begin{FOR}{i\=1 \TO 27}
		resultado.definiciones.siguientes[i] = NULL
	\end{FOR}
	resultado.definiciones.significado = NULL\\
	\RETURN resultado
	\end{algorithm}
	\textit{Complejidad:} 27 * O(1)=\textbf{O(1)}\\
	\textit{Aliasing:} No \\
	
	\begin{algorithm}{$i$DefinirDS}{\param{inout}{d}{estr}, \param{in}{c}{clave}, \param{in}{s}{significado}}{}
	Agregar(c, d.claves);\\
	Agregar(s, d.conjSiguientes);\\
	\VAR{punt\!: punteroAsignificado} \leftarrow \&s\\
	\VAR{actual\!: nodo}\\
	actual = d.definiciones;\\
		\begin{FOR}{i=1 \TO Longitud(c)}
			\begin{IF}{actual.siguientes[int(c[i])] == NULL}\\
				\VAR{nuevoNodo\!: nodo}\\
				nuevoNodo.significado = NULL\\
				\begin{FOR}{j=1 \TO 27}
					nuevoNodo.siguientes[j] = NULL
				\end{FOR}
				actual.siguientes[c[i]] = \&nuevoNodo
			\end{IF}\\
			actual = ({}^* actual).siguientes[int(c[i])];					
		\end{FOR}
	actual.significado = *punt
	\end{algorithm}	
	\textit{Complejidad:} |c| *(27*(O(1))) = O(|c|)\\
	\textit{Aliasing:} No \\	
	
	
	\begin{algorithm}{$i$Definido?DS}{\param{in}{d}{estr}, \param{in}{c}{clave}}{\param{}{resultado}{bool}}
	\VAR{ def \!: bool}\\
	\VAR{actual \!: nodo}\\
	\begin{IF}{(d.definiciones.siguientes[c[1]]) \neq NULL} \\
		actual={}^* (d.definiciones.siguientes[c[1]])\\
		def \leftarrow true \\
		\begin{FOR}{i=2 \TO Longitud(c)}
			\begin{IF}{(actual.siguientes[int(c[i])] \neq NULL$ $ \&\&$ $ def)}		\\
				actual = {}^* (actual.siguientes[int(c[i])])
			\ELSE \\
				def \leftarrow false
			\end{IF}
		\end{FOR}
		\begin{IF}{actual.significado \neq NULL}\\
			\RETURN def
		\ELSE \\
			def \leftarrow false
		\end{IF}
		\ELSE \\
			def \leftarrow false
		\end{IF} \\
	\RETURN def
	\end{algorithm}
	\textit{Complejidad:}\textbf{ O(|c|)}\\
	\textit{Aliasing:} No \\		
	
	
	\begin{algorithm}{$i$ObtenerDS}{\param{in}{d}{estr}, \param{in}{c}{clave}}{\param{}{resultado}{significado}}
	\VAR{actual\!: nodo}\\
	actual = d.definiciones\\
	\begin{FOR}{i=1 \TO Longitud(c)}
		actual = {}^* (actual.siguientes[int(c[i])])
	\end{FOR}
	\RETURN actual.significado
	\end{algorithm}
	\textit{Complejidad:}\textbf{ O(|c|)}\\
	\textit{Aliasing:} Si \\	
	
	\begin{algorithm}{$i$ClavesDS}{\param{in}{d}{estr}}{\param{}{resultado}{conj}}
	\RETURN d.claves
	\end{algorithm}
	\textit{Complejidad:}\textbf{ O(1)}\\
	\textit{Aliasing:} Si \\	
	
	\begin{algorithm}{$i$CopiarDS}{\param{in}{d}{diccionario}}{\param{}{resultado}{diccionario}}
	resultado.claves= Copiar(d.claves)\\
	resultado.conjSignificados= Copiar(d.conjSignificados)\\
	\begin{FOR}{i=1 \TO 27}
		\begin{IF}{d.siguientes[i]!=NULL}
			resultado.siguientes[i]={}^*iCopiarNodo(d.siguientes[i])
		\ELSE
			resultado.siguientes[i]=NULL
		\end{IF}
	\end{FOR}
	\RETURN resultado
	\end{algorithm}
	\textit{Complejidad:} O(\#(claves)*|clave\textsubscript{max}|)+O(\#(significados)*copiarSignificado)+27*O(copiarNodo) \\
	Donde \#(claves) == \#(significados) \\
	O(copiarNodo)=O(copiarSignificado) $\Rightarrow$ \\
	O(\#(claves) (O(|clave\textsubscript{max}|)+ O(copiarSignificado)))+O(copiarSignificado) = \\
	\textbf{O(\#(claves) (O(|clave\textsubscript{max}|)+ O(copiarSignificado)))}
	
	\textit{Aliasing:} No \\
	
	\begin{algorithm}{$i$CopiarNodo}{\param{in}{n}{puntero(nodo)}}{\param{}{resultado}{nodo}}
	\begin{IF}{\& n.significado!=NULL}
		resultado.significado \leftarrow Copiar(\& n.significado)
		\ELSE
		resultado.significado=NULL
	\end{IF} \\
	\begin{FOR}{i=1 \TO 27}
		\begin{IF}{\& n.siguientes[i]!=NULL}
			resultado.siguientes[i]={}^*iCopiarNodo(n.siguientes[i])
		\ELSE
			resultado.siguientes[i]=NULL
		\end{IF}
	\end{FOR}
	\RETURN resultado
	\end{algorithm}
	\textit{Complejidad:} 27*O(copiarSignificado) = \textbf{O(copiarSignificado})\\
	\textit{Aliasing:} No \\
	
\newpage